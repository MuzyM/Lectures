\documentclass[a4paper,12pt]{article}

%%% Работа с русским языком
\usepackage{cmap}					% поиск в PDF
\usepackage{mathtext} 				% русские буквы в формулах
\usepackage[T2A]{fontenc}			% кодировка
\usepackage[utf8]{inputenc}			% кодировка исходного текста
\usepackage[english,russian]{babel}	% локализация и переносы

%%% Дополнительная работа с математикой
\usepackage{amsfonts,amssymb,amsthm,mathtools} % AMS
\usepackage{amsmath}
\usepackage{icomma} % "Умная" запятая: $0,2$ --- число, $0, 2$ --- перечисление

%% Номера формул
%\mathtoolsset{showonlyrefs=true} % Показывать номера только у тех формул, на которые есть \eqref{} в тексте.

%% Шрифты
\usepackage{euscript}	 % Шрифт Евклид
\usepackage{mathrsfs} % Красивый матшрифт

%% Свои команды
\DeclareMathOperator{\sgn}{\mathop{sgn}}

%% Перенос знаков в формулах (по Львовскому)
\newcommand*{\hm}[1]{#1\nobreak\discretionary{}
{\hbox{$\mathsurround=0pt #1$}}{}}

%%% Работа с картинками
\usepackage{graphicx}  % Для вставки рисунков
\graphicspath{{images/}{images2/}}  % папки с картинками
\setlength\fboxsep{3pt} % Отступ рамки \fbox{} от рисунка
\setlength\fboxrule{1pt} % Толщина линий рамки \fbox{}
\usepackage{wrapfig} % Обтекание рисунков и таблиц текстом

%%% Работа с таблицами
%\usepackage{array,tabularx,tabulary,booktabs} % Дополнительная работа с таблицами
%\usepackage{longtable}  % Длинные таблицы
%\usepackage{multirow} % Слияние строк в таблице


%%% Заголовок
\author{Nikita Muzykantov}
\title{Матанализ. Семинар.}
\date{\today}

\begin{document}
\maketitle

\[ \lim_{x\to 0} \frac{a^x-1}{x} = \lim_{\log_a(1+y)\to 0}{\frac{y}{log_a (1+y)}} = \lim_{y\to 0} \frac{y\ln a}{\ln (y+1)} = \ln a
\]

\[ \lim_{x\to 0}\frac{a^x-b^x}{x} = \lim_{x\to 0}\left(\frac{a^x-1}{x}-\frac{b^x-1}{x}\right) = \ln a - \ln b = \ln \frac{a}{b}
\]

\[ \lim_{x\to 1} \frac{x^x-1}{x-1} = \lim_{x\to 1}\left(\frac{e^{x\ln x}}{x-1}\times \frac{x\ln x}{x\ln x}\right) = \lim_{x\to 1}\left(\frac{e^{x\ln x}}{x\ln x}\times \frac{x\ln x}{x-1}\right)
\]
Замена: $x=1+y$ 

\[ \lim_{x\to 1}\frac{\sin{\pi x^\alpha}}{\sin{\pi x^\beta}}=  \lim_{x\to 1}\frac{-\sin{\pi (x^\alpha -1)}}{-\sin{\pi (x^\beta -1)}} =
\lim_{x\to 1}\frac{x^\alpha -1}{x^\beta -1} = \]\\ \[
=\lim_{x\to 1}\frac{(x-1)(x^\alpha -1)+(x^\alpha -2)+\dots+1}{(x-1)(x^\beta -1)+(x^\beta -2)+\dots+1}=\frac{\alpha}{\beta}
\]

\section{Задачи на выделение главного члена}
1.Дано: $x\to 1; x^3-3x+2; c(x-1)^n$. Пусть $x=1+y$
\[ \lim_{x\to 1}\frac{x^3-3x+2}{(x-1)^n}=\lim_{y\to 0}\frac{1+3y+3y^2+y^3-3y-3+2}{y^n} = lim_{y\to 0}\frac{y^3+3y^2}{y^n}=
\]\\
\[lim_{y\to 0} \frac{y+3}{y^{n-2}}=3\]
\[x^3-3x+2=3(x-1)^2+O((x-1)^3)\]


2. Дано: $x\to 1; \sqrt[3]{1-\sqrt{x}}; c(x-1)^n$.

\[ \lim_{x\to 1}\frac{\sqrt[3]{1-\sqrt{x}}}{(x-1)^n}=\lim_{y\to 0}\sqrt[3]{\frac{(1-sqrt(x))(1+sqrt(x))}{(x-1)^{3n}(1+sqrt(x))}} = lim_{y\to 0}\frac{y^3+3y^2}{y^n}=
\]

\[=lim_{x\to 1}\frac{-1}{\sqrt[3]{}}\]


3.Дано: $x\to 1; \ln x; c(x-1)^n$.

\[ \lim_{x\to 1}\frac{\ln x}{(x-1)^n}=\lim_{y\to 0}\frac{\ln(1+y)}{y^n}=c=1
\]


4. Дано: $x\to 1; e^x-e; c(x-1)^n$.

\[ \lim_{x\to 1}\frac{e^x-e}{(x-1)^n}=\lim_{y\to 0}\frac{e^{1+y}-e}{y^n}=\lim_{y\to 0}e\frac{e^y-1}{y^n}\]


Ответ: $e^x-e=e(x-1) + o(x-1)$


5. \[\lim_{n\to \infty}\left(\frac{6n}{3n^2-1}\sin(n^2)+\frac{sqrt(n)}{sqrt(n+1)+sqrt(n)}\right)=
\]
\[= \lim_{n\to \infty}\dots\]


6. \[\lim_{x\to 0}\left(\frac{1+x\times 2^x}{1+x\times 3^x}\right)^{\frac{1}{x^2}}=\]
\[=\lim_{x\to 0}\left(1+\frac{x(2^x-3^x)}{1+x\times 3^x}\right)^{\frac{1}{x^2}}=
\lim_{x\to 0}\left(1+\frac{x(2^x-3^x)(1+\ln 2-1-x\ln 3)}{(1+x\times 3^x)(1+\ln 2-1-x\ln 3)}\right)^{\frac{1}{x^2}}=
\]
\[=\lim_{x\to 0}\left(1+\frac{x^2(2^x-3^x)\ln \frac{2}{3}}{(x\ln \frac{2}{3}(1+x\times 3^x))}\right)^{\frac{1}{x^2}\times \frac{\ln \frac{2}{3}}{\ln \frac{2}{3}}}=e^{\ln \frac{2}{3}}=\frac{2}{3}\]
Т.к. $\frac{a^x-1}{x} \rightarrow_{x\to 0} \ln a$ \\ \\
7. \[\lim_{x\to \frac{\pi}{4}} \frac{\sqrt[3]{\tg x}-1}{\log_{\frac{\pi}{4}}-1}= \]
\[=\lim_{y\to 0}\ln \frac{\pi}{4} \frac{\frac{4y}{\pi}}{\ln (\frac{4y}{\pi}+1)\times \frac{4y}{\pi}}\times \frac{\left(\sqrt[3]{\frac{2\sin y}{\cos y-\sin x}}-1\right)}{den}
\]

8. $\lim f(x)$

$x\to +-0$
\end{document}